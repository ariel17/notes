%        File: algoritmos_complejos.tex
%     Created: Sat Mar 29 08:00 PM 2014 A
% Last Change: Sat Mar 29 08:00 PM 2014 A
%
\documentclass[a4paper]{report}
\usepackage[spanish]{babel}
\usepackage[utf8]{inputenc}
\usepackage[]{hyperref}
\usepackage{listings}

\begin{document}

    \title{Algoritmos Complejos para Estructuras de Datos Avanzados}
    \author{Ariel Gerardo Ríos}
    \date{2014}
    \maketitle

    \tableofcontents

    \chapter{Presentación}

        \section{Profesores}

            \begin{itemize}
                \item Pablo Sznajaleder
                    \begin{tabular}{l@{\thinspace}l}
                        Twitter:    &\href{https://twitter.com/pabloszn}{@pabloszn}\\
                        Twitter:    &\href{https://twitter.com/thejavalistener}{@thejavalistener}\\
                        Facebook:   &\href{https://www.facebook.com/thejavalistener}{The Java Listener}\\
                        E-mail:     &\href{mailto:pablosz@gmail.com}{pablosz@gmail.com}
                    \end{tabular}

                \item Oscar Bruno
                    \begin{tabular}{l@{\thinspace}l}
                        Twitter:    &\href{https://twitter.com/arbruno}{@arbruno}\\
                        E-mail:     &\href{mailto:oscarrbruno@gmail.com}{oscarrbruno@gmail.com}
                    \end{tabular}
            \end{itemize}

        \section{Temas a ver en el curso}

            \begin{itemize}
                \item Recursividad
                \item Árboles
                    \begin{itemize}
                        \item Árbol binario de búsqueda
                        \item Árbol N-ario
                        \item AVL
                        \item B-Tree
                    \end{itemize}
                \item Complejidad
                \item Métodos de ordenamiento
                \item Estrategia algorítmica
                \item Grafos
            \end{itemize}

        \section{Ejercicios}

            \begin{enumerate}
                \item ¿Por qué tarda tanto una clase que calcula la serie de Fibonacci?
                \item ¿Cómo se podría resolver para que no tarde tanto?
                \item ¿Cuánto tiempo tardaría la ejecución si se pide un Fibonacci de 100? Estimar.
                \item Hacer un programa que realice todas las permutaciones de una cadena.
            \end{enumerate}

        \section{Soluciones}

            \begin{enumerate}
                \item Permutación de cadenas:
                    \lstinputlisting[label=permutacion,caption=Permutación de una cadena,language=java,basicstyle=\ttfamily, columns=flexible]{./ejercicios/permutation/src/ar/com/ariel17/permutation/Permutation.java}
            \end{enumerate}


\end{document}
