%        File: empresa-familiar.tex
%     Created: Mon Mar 10 04:00 AM 2014 A
% Last Change: Mon Mar 10 04:00 AM 2014 A
%
\documentclass[a4paper]{report}
\usepackage[utf8]{inputenc}
\begin{document}

  \title{Resumen del curso Empresa Familiar en Coursera}
  \author{Ariel Gerardo Ríos}
  \date{}
  \maketitle

  \chapter{Importancia y problemática de la empresa familiar}

  Las empresas familiares no son necesariamente empresas pequeñas, sino que las
  hay de gran tamaño. Ejemplo:

  \begin{itemize}
    \item The New York Times
    \item Michelin
    \item Ford
  \end{itemize}
  
  Las empresas familiares tienen un mejor rendimiento en el crecimiento de las
  utilidades y en el retorno a los accionistas. Esto sucede porque este tipo de
  empresas tienen ventajas frente a otros tipos de organizaciones:

  \begin{itemize}
    \item Visión de largo plazo: inversión de acciones pensando que el
          patrimonio va a mantenerse para las otras generaciones.
    \item Mayor compromiso de los miembros de la familia con el trabajo de la
          empresa.
    \item Responsabilidad en la región: no despiden a los empleados y colaboran
          con fundaciones para el beneficio de la comunidad.
    \item Liderazgo, cultura, valores y redes que comparte el empresario hacia
          hacia las otras generaciones son transmitidas hacia dentro de la
          compañía.
  \end{itemize}

  Las empresas familiares generan la mayor cantidad del Producto Interno de los
  países.


\end{document}


