%        File: ml.tex
%     Created: Sat Mar 08 09:00 AM 2014 A
% Last Change: Sat Mar 08 09:00 AM 2014 A
%
\documentclass[a4paper]{report}
\usepackage[utf8]{inputenc}
\begin{document}

  \title{Machine Learning course on Coursera summarization}
  \author{Ariel Gerardo Ríos}
  \date{}
  \maketitle

  \chapter{Begining}
    \section{Machine learning definition}
      \begin{itemize}
        \item Arthur Samuel (1959): Machine Learning: Field of study that gives
              computers the ability to learn without being explicitly
              programmed.
        \item Tom Mitchell (1998): Well-posed Learning Problem: A computer
              program is said to learn from experience E with respect to some
              task T and some performance measure P, if its performance on T,
              as measured by P, improves with experience E.
      \end{itemize}

    \section{Machine learning algorithms}
      \begin{itemize}
        \item To be seen on this course:
              \begin{itemize}
                \item Supervised learning
                \item Unsupervised learning
              \end{itemize}
        \item Others:
              \begin{itemize}
                \item Reinforcement learning
                \item Recommender systems
              \end{itemize}
      \end{itemize}
      Also talk about: Practical advice for applying learning algorithms.

  \section {Supervised learning}

    It assumes that the values given to learn are the right answers to the
    given question.

    \subsection{Regression problems}

      It tries to predict continuous valued output. Example: predict a house
      sell price.
      
    \subsection{Classification problems}

      It tries to assign a discrete value (classification)from a given output.
      Example: Cancer $\rightarrow$ \texttt{benign=0, malign=1}.


\end{document}
